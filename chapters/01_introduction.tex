Vue.js \parencite{vuejs_gh} is a popular progressive frontend framework for building user interfaces and single-page applications. Automatic test generation is a fascinating problem in Computer Science, especially for dynamically typed languages such as JavaScript. It presents a series of benefits, predominantly conserving of development time and allowing for more focus on the development of the main application. Interaction Diagrams can also aid the understand of \glspl{spa}, since in \glspl{spa} there is no clear navigation, rather contents are updated dynamically based on user input. 

The focus of this thesis lies on the implementation of an algorithm to generate Interaction Diagrams, as described by \textcite{zhang2019scenario}. The approach by \textcite{zhang2019scenario} is also  extended to also be able to model lists and complex objects. Further it is shown how the generated diagrams can be used in order to generate scenario templates in the Gherkin \gls{dsl}, which can be used in \gls{bdd}.


The thesis is structured as follows:

First fundamental technologies will be explained in Chapter \ref{chapter:fundamentals}. Then in Chapter \ref{chapter:concept} the concept of the algorithms will be described, followed by Chapter \ref{chapter:implementation}, which will delve deeper into some implementation details. Finally the application will be evaluated in Chapter \ref{chapter:testing_and_evaluation},
followed by the conclusion \ref{chapter:conclussion}.