1. zhang code, generated

2. example with lists
3. complex list(add, sub) with focus on updates

\section{Scenario}
\textcite{zhang2019scenario} use an example of a small \gls{spa} titled "Math for kids", which generates simple math addition problems and keeps track of a small statistic, which includes the number of correctly and incorrectly answered problems. The same application has been implemented and extended in Vue.js and will be used to display the capabities of the interaction diagram and scenario generation application.
\section{Math for Kids}

\begin{figure}[H]
    \centering
    \begin{subfigure}[b]{0.45\textwidth}
         \centering
         \includegraphics[width=0.45\textwidth]{images/math_for_kids_zhang.png}
         \caption{Math for Kids in AngularJS by \textcite{zhang2019scenario}}
         \label{fig:evaluation_math_kids_zhang}
    \end{subfigure}\hfill%
    \begin{subfigure}[b]{0.45\textwidth}
        \centering
        \includegraphics[width=0.45\textwidth]{images/math_for_kids_own.png}
        \caption{Own implementation of Math for Kids in Vue.js}
        \label{fig:evaluation_math_kids_own}
    \end{subfigure}\hfill%
\end{figure}

The source code can be found in \code{resources/test-files/test.vue} and also in \ref{appendix:math_kids_basic_source_code}.
\begin{figure}[H]
    \includegraphics[width=\textwidth]{images/diagram_own_math_kids.png}
     \caption{Math for Kids in Vue.js generated interaction diagram }
     \label{fig:math_for_kids_own_interaction_diagram}
\end{figure}

\begin{figure}[H]
    \includegraphics[width=\textwidth]{images/interaction_diagram_zhang.png}
     \caption{Math for Kids in AngularJS interaction diagram by \textcite{zhang2019scenario}}
     \label{fig:math_for_kids_zhang_interaction_diagram}
\end{figure}

When comparing the generated diagram\ref{fig:math_for_kids_own_interaction_diagram} to the original by \textcite{zhang2019scenario} \ref{fig:math_for_kids_zhang_interaction_diagram} there are some differences.

Due to different modelling there is a \code{this} vertex and connections from it to top level variables, which does not exist in the diagram by \textcite{zhang2019scenario}. In the generated diagram there is an edge of type 'event' between the \textit{answer} tag and \textit{answer} property. This is due to the way that two-way bindings were decided to be represented.

There are two differences between the \textit{add\_problem} node in the generated diagram and the one by \textit{zhang2019scenario}.
In \textcite{zhang2019scenario} there is an edge from
the \textit{add\_problem} method to the \textit{New Problem} tag, which is missing in the own generated version. This seems like a mistake in \parencite{zhang2019scenario}, since this edge should not exist.

The other difference is that in \parencite{zhang2019scenario} there is no edges from \textit{add\_problem} to \textit{right} (write relation), however there should be, 
inside \textit{add\_problem} \code{this.right = undefined}.

\begin{lstlisting}
l(created) -> a, b, answer, Check, Right, Wrong
l(answer) -> answer, Check
l(Check) -> Right, Wrong, Check, count_right, count_wrong
l(New Problem) -> a, b, answer, Check, Right, Wrong
\end{lstlisting}

The generated reaction based on the graph differ for $l(created)$, all other sets are exactly the same as the ones by \textcite{zhang2019scenario} (albeit in different order). This may stem from the fact that the edges of \textit{add\_problem} to \code{\$scope.right} is missing in \parencite{zhang2019scenario}, however it is later correctly factored in $l(New Problem)$.

\textcite{zhang2019scenario} define $l(add_problem())$ 
(the equivalent of $l(created)$) as $l(add_problem() = 
{a, b})$ where it should be 
$l(add_problem() = {Check, Right, Wrong, a, b, answer })$ since the answer property is updated based on the diagram. 

One could argue, that the version in \parencite{zhang2019scenario} is correct, since $answer$ is set to $undefined$, which is the same as the initial value of the variable, so it would not trigger an update as part of the \code{init} method. The interaction diagram generator does not perform this check. If $l(add_problem())$ were to be called later in the application (when the $New Problem$ button is clicked) it would indeed set a new value to $answer$, which is correctly reflected by \textcite{zhang2019scenario}.


\subsection{Scenarios}
Gherkin scenarios of up to 4 actions are generated, which can be seen in the following figures. The program outputs the scenarios as plain text to the console, but here they are displayed in a nicer way. The caption of each figure is an example of how a text could be generated based on the template scenario output by the application.

%TODO cite and explain  this in first section and ref from here
Some scenarios seem a bit repetitive, but it is up to \textit{The Three Amigos} - The product owner, tester and developer \parencite{cucumber_amigos} to decide which templates to discard. Since interaction diagrams model what \textit{might} be updated, it will probably be reasonable to discard some scenarios. The fact that everything which might get updated can also be leveraged in another way - negative criteria can be definied (assert tag/component $X$ did not change).

\begin{figure}[H]
    \centering
    \begin{subfigure}[b]{0.48\textwidth}
         \centering
         \includegraphics[width=0.48\textwidth]{images/scenarios_1.png}
         \caption{Scenario: initialization. When the application is created, then 'a' and 'b' should show random numbers and 'Check' should be disabled and 'Right' and Wrong should be invisible.}
    \end{subfigure}\hfill%
    \begin{subfigure}[b]{0.48\textwidth}
        \centering
        \includegraphics[width=0.48\textwidth]{images/scenarios_2.png}
        \caption{Scenario: typing an answer. Given the application has been created, when I type an answer then 'answer' should display it and 'Check' should be enabled.}
    \end{subfigure}\hfill%
    \begin{subfigure}[b]{0.48\textwidth}
        \centering
        \includegraphics[width=0.48\textwidth]{images/scenarios_3.png}
        \caption{Scenario: clicking on check without typing an answer. Given the application has been created, when I click 'Check' then 'Right' and 'Wrong' should be invisible and 'count\_right' and 'count\_wrong' should have the same values and 'Check' should be disabled. }
   \end{subfigure}\hfill%
   \begin{subfigure}[b]{0.48\textwidth}
    \centering
    \includegraphics[width=0.48\textwidth]{images/scenarios_4.png}
    \caption{Scenario: obtaining a new problem at application start. Given the application has been created, when I click 'New Problem' then 'a' and 'b' should have random values and 'answer' should be empty and 'Check' should be disabled and 'Right' and 'Wrong' should be invisible.}
\end{subfigure}\hfill%
\end{figure}
\begin{figure}[H]\ContinuedFloat
    \centering
\begin{subfigure}[b]{0.48\textwidth}
    \centering
    \includegraphics[width=0.48\textwidth]{images/scenarios_5.png}
    \caption{Scenario: checking my answer. Given the application has been created and 'answer' contains my answer, when I click 'Check' then 'Right' should be visible if the answer was right 'Wrong' should be visible if the answer was wrong and 'Check' should be disabled and 'count\_right' should be incremented by one if the answer was right and 'count\_wrong' should be incremented by one if the answer did not change. }
\end{subfigure}\hfill%
\begin{subfigure}[b]{0.48\textwidth}
    \centering
    \includegraphics[width=0.48\textwidth]{images/scenarios_6.png}
    \caption{Scenario: typing an answer to a new problem. Given the application has been created and a new problem was obtained, when I type an answer then 'answer' should display it and 'Check' should be enabled.}
\end{subfigure}\hfill%
\begin{subfigure}[b]{0.48\textwidth}
    \centering
    \includegraphics[width=0.48\textwidth]{images/scenarios_7.png}
    \caption{Scenario: requesting a new problem and clicking on check without answering it. Given the application has been created and I requested a new problem, when I click 'Check' then 'Right' and 'Wrong' should be invisible and 'count\_right' and 'count\_wrong' should have the same values and 'Check' should be disabled. }
\end{subfigure}\hfill%
\begin{subfigure}[b]{0.48\textwidth}
    \centering
    \includegraphics[width=0.48\textwidth]{images/scenarios_8.png}
    \caption{Scenario: requesting a new problem, answering it and checking my answer. Given the application has been created and I requested a new problem and answered it, when I click 'Check' then 'Right' should be visible if the answer was right 'Wrong' should be visible if the answer was wrong and 'Check' should be disabled and 'count\_right' should be incremented by one if the answer was right and 'count\_wrong' should be incremented by one if the answer did not change. }
\end{subfigure}\hfill%
\end{figure}




\section{Diagram with list}
which can be found in \code{resources/test-files/list.vue}
\begin{figure}[H]
    \includegraphics[width=\textwidth]{images/diagram_list.png}
     \caption{TODO}
     \label{fig:diagram_list}
\end{figure}

% \section{Diagram Addition and Substitution}
% \begin{figure}[H]
%     \includegraphics[width=\textwidth]{images/diagram_list_add_sub.png}
%      \caption{TODO }
%      \label{fig:diagram_test_add_sub}
% \end{figure}


%TODO link to https://github.com/KarakoA/vuejs-example and say examples can be deployed there