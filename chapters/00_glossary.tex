%LateX docs:
%https://www.overleaf.com/learn/latex/glossaries

%http://ftp.rrzn.uni-hannover.de/pub/mirror/tex-archive/macros/latex/contrib/glossaries/glossariesbegin.pdf



\newglossaryentry{visitor}
{
    name={Visitor},
    description={Is a design pattern used to separate an algorithm from the object structure on which it operates}
}

\newglossaryentry{directed_graph}
{
    name={Directed Graph},
    description={Is a graph that is made up of a set of vertices connected by edges, where the edges have a direction associated with them}
}
\newglossaryentry{compound_graph}{
    name={Compound Graph},
    description={Is a graph that can have sub graphs}
}


\newglossaryentry{abstract_syntax_tree}
{
    name={Abstract Syntax Tree},
    description={Is a tree representation of the abstract syntactic structure of source code written in a programming language. Each node of the tree denotes a construct occurring in the source code}
}

\newglossaryentry{ts}
{
    name={TypeScript},
    description={Is a programming language, which is a strict syntactical superset of JavaScript and adds optional static typing to the language.}
}

\newglossaryentry{databinding}
{
    name={data binding},
    description={Is a technique, that enables synchronization of data between consumers and providers}
}


\newacronym{dsl}{DSL}{domain-specific language}

\newacronym{bdd}{BDD}{behavior-driven Development}

\newacronym{ide}{IDE}{Integrated development environment}

\newacronym{mvvm}{MVVM}{Model-View-ViewModel}

\newglossaryentry{ast}{type=\acronymtype, name={AST}, description={Abstract syntax tree},see=[Glossary:]{abstract_syntax_tree}}

%\newacronym{ast}{AST}{}

\newacronym{ui}{UI}{user interface}

\newacronym{spa}{SPA}{single-page application}

\newacronym{guid}{GUID}{Globally Unique Identifier}

\newacronym{dom}{DOM}{Document Object Model}

%TODO see if needed 
\glsaddall