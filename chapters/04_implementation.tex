\section{Project Structure and Overview}
The application is written in TypeScript(excluding the ESLint visitors, which have to be in JavaScript) using NPM as a package manager and Node.js as a runtime environment. 

Notable dependencies are TypeScript, for stricter syntax and types, lodash\parencite{lodash} for enrichening of collections, graphlib\parencite{graphlib} for the interaction diagrams graph,  babel\parencite{babel} as a transcompiler, eslint\parencite{eslintMainPage}, Estree\parencite{estreeASTSpec} and eslint-plugin-vue\parencite{eslint_vue_parser} for parsing Vue.js code, d3-graphviz\parencite{graph_viz} in combination with light-server\parencite{light_server} for visualization of the generated interaction diagrams. The full list of dependencies can be found in the \code{package.json} of the project.

The application source code is linted using ESLint, unit tests written in jest (a total of 109 tests cases) and also includes simplistic regression tests. Git is used for version control and also a CI/CD pipelines is setup, which on each commit runs the tests and regression tests and also ensure there are no type errors, as reported by TypeScript, and no linting errors, as reported by ESLint. 
A separate minimal, Vue.js project is included \parencite{KarakoA2021Feb} in which the test \gls{spa} applications can be executed in viewed in a browser.

\subsection{Project structure}
The main source files of the project and their tests are included in the \code{src} directory and structured in several packages, each corresponding to a step in the process, except common, which is shared among all steps. Each will be described in more detail in the following sections.
Each package includes a models directory, which includes the data types defined and used in this section.
The \code{web} directory contains code used to view the resulting diagram in the browser.
The \code{scripts} directory includes helper bash scripts, \code{results} holds snapshots of the results throughout development (with the latest being the current) and \code{resources} hold various additional files needed.
\begin{lstlisting}[basicstyle=\tiny]
  root:.
  |-- resources
  |   |-- output
  |
  |-- results
  |
  |-- scripts
  |
  |-- src
  |   |-- common
  |   |   |-- models
  |   |
  |   |-- main.ts
  |   |
  |   |-- generator
  |   |   |-- models
  |   |
  |   |-- parsing
  |   |   |-- builders
  |   |   |-- models
  |   |   |-- visitors
  |   |
  |   |-- scenarios
  |
  |-- web
\end{lstlisting}

\section{Parsing Vue.js}

Instead of implementing a parser, which directly outputs the simplified Vue.js \gls{ast} described in \ref{ast} the capabilities of \textcite{eslintMainPage} 
and \parencite{eslint_vue_parser} were used. 
The source files, which handle the parsing reside in the \code{parsing} directory.
The \code{ESLinter} class provides a wrapper around the Node.js API of ESlint. Custom visitors are implemented in order to extract the necessary nodes from the \gls{ast} of ESLint. Each visitor has a matching file in \code{models}, which holds the models specific to that visitor, and a \code{builder}, which keeps track of the visited nodes and builds the result data type. There are a total of three visitors - one for top level properties, another for bindings and the last one for method definitions, computed properties and the created method.
 
\subsection{Common Data Types}
Below are the common data types used by all visitors.
\begin{lstlisting}[language=JavaScript]
export type Identifier = This | NameIdentifier | NumericIndex 
| GenericIndex;

interface BaseIdentifier {
  readonly name: string;
}
export interface This extends BaseIdentifier {
  name: "this";
  discriminator: IdentifierType.THIS;
}
export interface NumericIndex extends BaseIdentifier {
  discriminator: IdentifierType.NUMERIC_INDEX;
}
export interface GenericIndex extends BaseIdentifier {
  discriminator: IdentifierType.GENERIC_INDEX;
}
export interface NameIdentifier extends BaseIdentifier {
  discriminator: IdentifierType.NAME_IDENTIFIER;
}
\end{lstlisting}

\begin{lstlisting}[language=JavaScript]
export type Entity = Method | Property;

export interface Property {
  id: Identifiers;
  discriminator: EntityType.PROPERTY;
}
export interface Method {
  id: Identifiers;
  args: ReadonlyArray<Entity>;
  discriminator: EntityType.METHOD;
}
\end{lstlisting}
Discriminators are used to be able to differentiate between the types using type guards. The enums themselves are omitted here (\code{IdentifierType}, \code{EntityType}). The definitions here are not exactly the same as in the \gls{ast} \ref{ast} - some constraints are omitted, such as method names having to end on a a \code{NameIdentifier}. This will be given, since the parsed code would be invalid JavaScript otherwise.
\subsection{Top Level Properties}
The result of the top level properties has the following data type.
\begin{lstlisting}[language=JavaScript]
export type TopLevelProperties = Array<Property>;

export interface TopLevelPropertiesResult {
  topLevel: TopLevelProperties;
}
\end{lstlisting}

The top level properties visitor is the simplest of all since it only reacts to the top level \code{data} node inside the \code{script} object  of the Vue.js \gls{spa}, which is a \code{ObjectExpression} \ref{eslint:object_expression}. It can be selected via the following selector:

%TODO lang
\label{impl:top_level_select}
\begin{lstlisting}[language=JavaScript]
"ExportDefaultDeclaration > ObjectExpression > 
Property[key.name = data] ReturnStatement > ObjectExpression"(node){
  ...
}
\end{lstlisting}
In natural language the selector reads: "Select ObjectExpression nodes, which have a direct parent ReturnStatement, that has an indirect parent Property with a property \code{key.name} equal to \code{data} and a direct parent ObjectExpression with a direct parent ExportDefaultDeclaration.

For each of the properties \ref{eslint:property} of the \code{ObjectExpression} the name of the key (identifier) is stored. If the property is an object (value of \code{ObjectExpression} \ref{eslint:object_expression}) it is concatenated with the previously obtained key. Finally all obtained properties are prefixed with 'this'.

\subsection{Bindings}
The result of the bindings visitor has the following data type.
\begin{lstlisting}[language=JavaScript]
  export enum BindingType {
    EVENT = "event", ONE_WAY = "one-way", TWO_WAY = "two-way",
  }
  
  export interface Tag {
    id: string;
    loc: Location;
    name: string;
    position?: string;
  }
  
  export interface BindingValue {
    item: Entity;
    bindingType: BindingType;
  }
  
  export type Binding = { tag: Tag; values: BindingValue[] };
  export interface BindingsResult {
    bindings: Binding[];
  }
\end{lstlisting}

The ESlint \gls{ast} nodes, which are interesting when parsing the bindings are \code{VElement} \ref{eslint:velement}, 
\code{Identifier} \ref{eslint:identifier}, \code{MemberExpression} \ref{eslint:member_expression} and \code{CallExpression} \ref{eslint:call_expression}. 

A \code{identifier} \ref{ast}, abstracted
in \code{common/identifier.ts}  can be a single \code{Identifier} or a \code{MemberExpression}, which can contain other \code{Identifier} nodes or \code{MemberExpression} nodes. Property Identifiers are extracted by finding the root \code{MemberExpression} or \code{Identifier} and traversing it. It is easy to determine if a \code{MemberExpression} or \code{Identifier} is the root - its parent is anything but a \code{MemberExpression}.

A \code{CallExpression} contains information about the name of the method and the arguments it has been called with, both in the form of nested \code{MemberExpression} and \code{CallExpression} nodes. Once again, only the root \code{CallExpression} node is extracted and converted to a \astnode{calledMethod} \ref{ast}, abstracted in the \code{Method} interface in \code{shared.ts}.

\code{VElement} nodes represent any HTML tag, matching a \astnode{tag} \ref{ast} abstracted in the \code{Tag} interface in code{template-bindings.ts} and contain information about the location of the tag and potentially a \code{VText} \ref{eslint:vtext} node, which will be set as its name if it is present. If not present, the name of the tag is equal to the name of the first binding. Therefore, information about tags is extracted once a \code{VElement} is exited, since all bindings will be known at this point.

Further, the binding type has to be determined. This can be extracted based on the \code{VAttribute} \ref{eslint:VAttribute}. Event bindings have a \code{VAttribute} with a \code{key.name.name} equal to 'on', two-way bindings equal to 'model' and everything else is interpreted as one-way bindings. This includes moustache statements, \code{v-bind}, \code{v-if} bindings and all other except \code{v-for} statements. This filter be achieved via the powerful \code{:not} in combination with \code{:matches} selectors: 
%TODO lang
\begin{lstlisting}[language=JavaScript]
:not(:matches(
  VAttribute[key.name.name=on], 
  VAttribute[key.name.name=model],
  VAttribute[key.argument.name=key],
  VAttribute[key.name.name=for]))
\end{lstlisting}

With all the above, for example to match all two-way bindings and pass them on to the builder can be done via 
\begin{lstlisting}[language=JavaScript]
"VAttribute[key.name.name=model] > VExpressionContainer
:matches(MemberExpression, Identifier, CallExpression)"(
  node) {
  if (utils.isRootNameOrCallExpression(node) && 
  utils.notArgument(node))
    builder.identifierOrExpressionNew(node, BindingType.TWO_WAY);
},
\end{lstlisting}

%TODO note why not args - method knows exactly what part

Bindings also need to substitute \code{v-for} statements. This is done by substituting the left side of the \code{v-for} statement with its right side and a generic index in all bindings that use it.

\subsection{Method Definitions}

\begin{lstlisting}[language=JavaScript]
export interface MethodDefinition {
  id: Identifiers;
  args: ReadonlyArray<Property>;
  reads: ReadonlyArray<Property>;
  writes: ReadonlyArray<Property>;
  calls: ReadonlyArray<Method>;
}
export type MethodDefinitions = Array<MethodDefinition>;

export interface MethodsResult {
  init?: MethodDefinition;
  computed: MethodDefinitions;
  methods: MethodDefinitions;
}
\end{lstlisting}
All method definition like structures (computed properties, created) and methods are parsed by the visitor defined in \code{methods.js}.
Analogous to how the top level \code{data} object is selected \ref{impl:top_level_select}, the \code{methods}, \code{created} and \code{computed} objects can be selected. 
The name of the method including its arguments can be extracted from by \code{Property[value.type=FunctionExpression]} nodes. Using this information, one can have three selectors, one of each type, to determine what is being defined. For example for regular methods:
 
\begin{lstlisting}[language=JavaScript]
"ExportDefaultDeclaration > ObjectExpression > Property
[key.name = methods] Property[value.type=FunctionExpression]"(node) {
  builder.newMethod(node, MethodType.METHOD);
},
\end{lstlisting}
Further the properties read, written and methods called need to be extracted. Methods called can be obtained by selecting \code{CallExpression} nodes. Properties written to can be obtained from the left side of a \code{AssignmentExpression} \ref{eslint:AssignmentExpression}.
There does not seem be an easy way to select all properties read from. Therefore all accessed identifiers are first stored and everything except reads, that can have an identifier (object properties, variable declarations, variables written to and names of called methods) is subtracted from it, in order to obtain the variables that the method reads from.

The following code can be used to obtain all variables written to by the current method-like in scope. 
\begin{lstlisting}[language=JavaScript]
"ExportDefaultDeclaration > ObjectExpression > 
:matches(Property[key.name = methods], Property[key.name = created],
Property[key.name = computed]) AssignmentExpression"(node) {
  builder.identifierOrExpressionNew(node.left, AccessType.WRITES);
},
\end{lstlisting}
\subsection{Output}
Combining all of the above, the following data structure is output.
\begin{lstlisting}[language=JavaScript]
export class Result {
  fileName: string;
  topLevel: TopLevelPropertiesResult;
  methods: MethodsResult;
  bindings: BindingsResult;
  ...
}
\end{lstlisting}
\label{impl:result}
\section{Interaction Diagram Generation}
The generation of the interaction diagram graph from the result class from \ref{impl:result} is done in the \code{Transformer} class. 

The resolution of methods is abstracted in the \code{MethodResolver} class. It produces a 
\code{ResolvedMethodDefinition} for each called method in bindings and the initial method. In order to prevent duplicate resolution of methods and wasting of resources a \code{MethodCache} is introduced. The \code{Transformer} does not use the \code{MethodResolver} directly, but instead accesses it via the \code{MethodCache}. The cache includes directly called (bound to) and indirectly called (calls of methods), for each of which vertices will have to be created. 
Each \code{ResolvedMethodDefinition} has the following data type:
\begin{lstlisting}[language=JavaScript]
export enum GeneralisedArgument {
  METHOD = "method",OTHER = "other",
}
export type ResolvedArgument =
  | Property | GeneralisedArgument.METHOD | GeneralisedArgument.OTHER;
export interface ResolvedMethodDefinition {
  id: Identifiers;
  args: ReadonlyArray<ResolvedArgument>;
  reads: ReadonlyArray<Property>;
  writes: ReadonlyArray<Property>;
  calls: ReadonlyArray<CalledMethod>;
}
export interface CalledMethod {
  id: Identifiers;
  args: ReadonlyArray<ResolvedArgument>;
}
\end{lstlisting}


As the underlying structure for the graph graphlib is used and wrapped in an own class - \code{ExtendedGraph}. It creates vertices on a 'create if not exist' basis by first looking up to see if the vertex exists in the graph, and if it does, does not add it again. Presence of edges is not checked, if an edge is added again, the previous one is simply overwritten. There can only be one edge per direction between two nodes, since no multigraph is used. Nodes and Edges in the graph have the following structure, as specified in \ref{concept:interaction_diagram_structure}:

\begin{lstlisting}[language=JavaScript]
export enum EdgeType {
  SIMPLE = "simple", EVENT = "event", CALLS = "calls",
}
export interface Edge {
  source: Node;
  sink: Node;
  label: EdgeType;
}
\end{lstlisting}
\begin{lstlisting}[language=JavaScript]
export enum NodeType {
  TAG = "tag",  DATA = "data", METHOD = "method", INIT = "init",
}

export type Node = TagNode | DataNode | MethodNode | InitNode;
interface BaseNode {
  id: string;
  name: string;
}
export interface TagNode extends BaseNode {
  loc: Location;
  discriminator: NodeType.TAG;
}
export interface MethodNode extends BaseNode {
  discriminator: NodeType.METHOD;
}
export interface InitNode extends BaseNode {
  discriminator: NodeType.INIT;
}
export interface DataNode extends BaseNode {
  parent?: string;
  type: IdentifierType;
  discriminator: NodeType.DATA;
}
\end{lstlisting}


The algorithm for generating the interaction diagrams graph as described in \ref{concept:algorithm_create_diagrams}
is implemented in the \code{Transformer} class.

First the init method is resolved by querying the \code{MethodCache} and afterwards each of the bindings. The order is not important. If the bindings are properties, the vertices for them are created directly. If they are methods or computed properties, they are resolved by the \code{MethodCache} and the correct edges based on the binding type created. The above is done in the \code{addInit()} and \code{addBindings()} methods.

At this stage no vertices are created for the reads, writes, and calls properties of the resolved methods. 
This happens in the \code{addIndirectlyCalledMethods()} method, after all bindings and the init method have been resolved by taking all methods stored in the \code{MethodCache} and creating the appropriate vertices.

Lastly in the \code{addEdgesForLists} method additional edges will be added for the \textit{all} vertices of properties of elements inside lists.

\section{Scenario Generation}
The scenario generation is implemented in the \code{scenario.ts} file and currently outputs the obtained scenarios, tags, nodes react to and the gherkin scenarios templates to the console. The scenarios are generated exactly as described in \ref{concept:scenario_generation}

\section{Display of Interaction Diagram}
In order to display the interaction diagrams it is displayed on a single html page, utilizing the d3-graphviz \parencite{graph_viz} library. 

Event edges are displayed as thick bold lines, calls edges are displayed as solid lines and all other as dashed lines.

Html tags are separated from all other nodes in a subgraph in order to be able to more easily identify them. In addition to their name, they also include the line(s) on which they are in the source code as a subline. The initial function has a grey background so it can be identified right away.
A button to download the displayed diagram as a \code{.png} image is also provided in the top left.

\section{Usage}
The application can be started using
%lang bash
\begin{lstlisting}[style=bash]
npm run generate -- [file to parse] [output graph path] [depth]
\end{lstlisting}
where file to parse is the path to the \code{.vue} file to be parsed and output graph path is the path to which the output interaction diagram graph will be written (JSON object). Depth is the factor $k$ in \ref{concept:scenario_generation}.
A web server can be started in order to view the results in a browser at \code{localhost:8000} via
%lang bash
\begin{lstlisting}[style=bash]
npm run serve
\end{lstlisting}
In order to create a snapshot for each file in \code{resources/test-files}.
%lang bash
\begin{lstlisting}[style=bash]
npm run create-results  
\end{lstlisting}
Can be used. This snapshot includes the output interaction diagram graph as \code{data.json}, the generated scenarios as a \code{scenario.txt}, \code{index.html}, which can be used to display the interaction diagram and a \code{.vue} file for which all of the above was generated.

All snapshots can be viewed using
%lang bash
\begin{lstlisting}[style=bash]
npm run results 
\end{lstlisting}
and navigating to \code{localhost:8001}

%TODO TS in  glossary?